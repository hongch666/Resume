% !TEX program = xelatex

\documentclass{resume}
\usepackage{graphicx}
\usepackage{tabu}
\usepackage{multirow}
\usepackage[utf8]{inputenc}  
\usepackage{zh_CN-Adobefonts_external} % 
\usepackage{linespacing_fix} % disable extra space before next section
\usepackage{cite}
\usepackage{subfig}   %并排图片宏包
\usepackage{float}    %图片浮动位置

% 减小页面边距
\usepackage{geometry}
\geometry{
  a4paper,
  left=0.6in,
  right=0.6in,
  top=0.5in,
  bottom=0.4in,
  nohead
}

\begin{document}
\pagenumbering{gobble} % suppress displaying page number

\begin{minipage}{0.8\textwidth}
 \fontsize{11pt}{0}{
  姓名:\textbf{褚世超} \hspace{5.05cm} 年龄:\textbf{21}  \\
  性别:\textbf{男} \hspace{5.85cm} 求职意向:\textbf{后端开发}  \\
  联系电话:\textbf{13642563270} \hspace{3.35cm}  邮箱:\textbf{hongch666@gmail.com}
 }
\end{minipage}
\begin{minipage}{0.2\textwidth}
\centering
\vspace{-0.2in}
\hspace{-0.4cm}
\includegraphics[height=1.0in,]{pic/pic.jpg}
\end{minipage} % 引入个人信息文件
% \begin{minipage}{0.8\textwidth}
%  \fontsize{11pt}{0}{
%   姓名:\textbf{褚世超} \hspace{3.05cm} 年龄:\textbf{21}  \\
%   性别:\textbf{男} \hspace{3.85cm} 求职意向:\textbf{后端开发}  \\
%   联系电话:\textbf{13642563270} \hspace{1.35cm}  邮箱:\textbf{hongch666@gmail.com}
%  }
% \end{minipage}

% \begin{minipage}{0.2\textwidth}
% \centering
% \vspace{-0.4in} % 向上移动图片
% \includegraphics[height=1.0in,]{pic/pic.jpg}
% \end{minipage}

\section{教育背景}
\begin{normalsize}
  \begin{itemize}[parsep=0.5ex]
    \item {2022.09 -- 至今}\hspace{0.8cm}华南师范大学(211)  \hspace{0.8cm} 计算机科学与技术 \hspace{0.8cm} 学士 \hspace{0.8cm} GPA:3.64(前15\%)
  \end{itemize}
\end{normalsize} % 引入教育背景文件

\section{专业技能}
\begin{normalsize}
  \begin{itemize}
    \item \textbf{数据库技术}:熟悉\textbf{MySQL}设计与优化,掌握事务和索引优化;熟悉\textbf{Redis}、\textbf{MongoDB}等\textbf{NoSQL}应用,能根据需求选择合适的数据存储;有\textbf{ElasticSearch}搜索引擎实践经验,能进行分词和查询优化,提升检索效率。
\item \textbf{消息队列}:熟练使用\textbf{RabbitMQ}实现\textbf{异步解耦}与\textbf{高可用消息系统},掌握\textbf{消息持久化}、\textbf{延迟与死信队列}等特性;了解\textbf{Kafka}分布式架构,有高并发消息流处理与管道搭建经验。
    % \item \textbf{Java后端}:熟悉Java基础、集合、JVM、并发编程与垃圾回收,掌握Spring生态(SpringBoot、SpringMVC)、MyBatis/MyBatisPlus及微服务(SpringCloud、Nacos、OpenFeign等),具备前后端分离与异步高并发处理经验,了解本地大模型部署与SpringAI集成。
    \item \textbf{Java与Spring开发}:熟悉Java基础语法、集合框架、JVM内存模型及并发编程,理解面向对象设计原则;掌握Maven构建、Spring生态(SpringBoot、SpringMVC等)及MyBatis/MyBatisPlus,具备RESTful接口与前后端分离开发能力,理解依赖注入(DI)与AOP机制。
\item \textbf{SpringCloud微服务开发}:熟悉SpringCloud微服务架构,熟悉Nacos服务注册与发现、Spring Cloud LoadBalancer负载均衡、OpenFeign服务调用、Sentinel服务保护、Gateway服务网关。
\item \textbf{SpringAI与大模型部署}:具备本地大模型部署经验,能够使用Ollama部署大模型并结合SpringAI构建AI应用;熟悉数据结构与算法,了解传统机器学习及深度学习算法。
    % \item \textbf{Go后端}:熟悉Go语言基础语法,掌握Goroutine与Channel并发模型;具备基于Gin框架的 RESTful接口开发能力,能够使用GORM进行数据库操作,完成高性能Web服务的开发与部署。
    \item \textbf{Go基础}:熟悉Go语言基础语法,理解类型系统、切片、映射、接口等核心特性,掌握Goroutine与Channel并发模型,具备高并发程序设计能力。
\item \textbf{Gin开发}:具备基于Gin框架的RESTful接口开发能力,能够进行路由设计、中间件编写和接口测试。
\item \textbf{Gorm数据库}:熟练使用GORM进行数据库建模与操作,能够实现复杂查询、事务控制和性能优化,支持高性能Web服务的开发与部署。
    % \item \textbf{Node.js后端}:熟悉Node.js及主流框架(如Express、NestJS),具备模块化后端开发能力,能够使用Mongoose操作MongoDB,熟练用TypeORM构建关系型数据模型并支持事务控制。
    \item \textbf{Node.js基础}:熟悉Node.js运行机制,理解事件循环、异步I/O、回调、Promise与async/await等异步编程模型,能够编写高效、可维护的后端服务。
\item \textbf{NPM模块化与包管理}:掌握CommonJS与ESM模块规范,熟悉npm/yarn包管理工具,能够合理组织项目结构与依赖管理。
\item \textbf{Express框架开发}:熟练使用Express进行RESTful接口开发,具备路由设计、参数校验、错误处理与中间件编写能力,能够实现会话管理与权限控制。
\item \textbf{NestJS框架开发}:熟悉NestJS的模块化架构、依赖注入(DI)、装饰器与中间件机制,能够基于TypeScript开发高可维护性和可扩展性的企业级后端服务。
\item \textbf{Mongoose数据建模}:熟悉Mongoose的Schema定义、模型关系与基本数据校验,能够进行常见文档结构设计。
\item \textbf{TypeORM数据库操作}:熟悉TypeORM实体定义、关系映射与事务管理,能够高效构建和操作关系型数据库模型。
    % \item \textbf{Python后端}:熟悉Python语言基础与异步编程,具备基于FastAPI框架的RESTful接口开发能力,能够使用SQLAlchemy ORM进行数据库操作,完成高性能Web服务的开发与部署;能够利用Python的高效库(如Pandas、NumPy等)进行数据处理、分析与可视化。
    \item \textbf{Python基础}:熟悉Python语言基础语法,理解数据类型、函数、面向对象编程等核心特性,掌握异步编程模型。
\item \textbf{FastAPI开发}:具备基于FastAPI框架的RESTful接口开发能力,能够进行路由设计、中间件编写和接口测试。
\item \textbf{SQLAlchemy数据库操作}:熟练使用SQLAlchemy ORM进行数据库建模与操作,能够实现复杂查询、事务控制和性能优化,支持高性能Web服务的开发与部署。
\item \textbf{数据分析与可视化}:能够利用Python的高效库(如Pandas、NumPy等)进行数据处理、分析与可视化。
    \item \textbf{实时通信}:熟悉\textbf{WebSocket},能在\textbf{Spring}、\textbf{Gin}、\textbf{NestJS}与\textbf{FastAPI}上实现实时推送、连接管理与跨实例广播。
\item \textbf{AI平台集成}:熟悉\textbf{Coze AI平台}的\textbf{API接入与集成},掌握\textbf{企业知识库构建}、\textbf{文档上传}、\textbf{智能问答}等功能开发,具备\textbf{AI助手}系统后端开发经验。
\item \textbf{前端技术}:熟悉\textbf{Javascript}/\textbf{Typescript},掌握\textbf{Vue.js}核心概念与\textbf{Element UI}组件库,具备独立构建和维护前端项目能力;了解\textbf{React框架}与\textbf{Ant Design}组件库,具备多框架技术选型与开发经验。
\item \textbf{开发与运维环境}:熟悉使用\textbf{Git}进行\textbf{版本控制}的流程;熟悉\textbf{Linux操作系统}的基本命令,能够在\textbf{Linux环境}下进行开发和部署;熟悉\textbf{Docker容器化部署},能够进行高效服务集群的搭建。
    
  \end{itemize}
\end{normalsize}

\section{实习经历}
\datedsubsection{\textbf{广州拓扑网络技术有限公司}\hspace{2.5cm}后端开发实习生}{2025.05 -- 2025.08}
\fontsize{11pt}{0}{\textbf{实习描述}:参与基于NestJS的企业级内部管理平台开发,负责RPA机器人管理、AI知识库、逆向物流自动化等模块后端开发,集成影刀RPA、Coze AI、钉钉开放平台,实现业务流程自动化与智能化管理。}
  \begin{normalsize}
    \begin{itemize}
    \item \textbf{核心职责}:
    \setlength{\itemindent}{1em} % 子项缩进调整
      \item[$\circ$] \textbf{后端框架与接口开发}:基于NestJS实现机器人、任务、调度、逆向物流等模块接口,支持影刀RPA机器人生命周期及逆向物流等业务自动化,集成Shopify GraphQL API,实现订单数据实时查询与同步。
      \item[$\circ$] \textbf{Coze AI知识库集成}:接入Coze AI平台,构建企业知识库管理系统,实现文档自动上传、知识检索、智能问答等功能。
      \item[$\circ$] \textbf{钉钉与JWT统一认证}:对接钉钉开放平台,支持钉钉扫码登录与消息推送,结合JWT实现统一身份认证与权限控制,日均支撑500+用户登录与消息通知,保障系统安全与高效协同。
      \item[$\circ$] \textbf{非阻塞I/O}:利用Node.js异步事件驱动与非阻塞I/O模型,在任务高并发调度与状态修改场景中保持系统高吞吐与低延迟,稳定支撑千级并发访问请求。
      \item[$\circ$] \textbf{数据库框架}:使用TypeORM构建数据模型与多表关联关系,设计机器人、任务、用户等核心实体映射,通过Repository模式封装数据访问层,优化复杂查询逻辑与索引策略,提升性能约40\%。
      \item[$\circ$] \textbf{MongoDB日志记录}:使用MongoDB结合Mongoose建模,记录系统关键业务操作与用户行为日志,支持日均2万+操作日志高效写入,突破关系型数据库字段限制。
      \item[$\circ$] \textbf{Redis缓存}:实现机器人状态缓存和任务缓存,接口查询响应时间由原500ms降低至20ms,支撑10万级高并发状态请求。
      \item[$\circ$] \textbf{RabbitMQ异步解耦}:集成RabbitMQ,实现任务异步投递与空闲机器人动态分配,系统日均调度任务数超5000条,任务分发延迟控制在200ms内。
    \setlength{\itemindent}{0em} % 子项缩进调整
    \item \textbf{相关技术}: NestJS/TypeORM/MySQL/Redis/MongoDB/RabbitMQ/Coze AI/钉钉开放平台
    \end{itemize}
  \end{normalsize}



\section{项目经历}

\subsection{\textbf{校园餐厅智能化场景下的外卖点单和管理平台} \hspace{2cm}{2025.01 -- 2025.04}}
\fontsize{11pt}{0}{\textbf{项目描述}:面向高校食堂数字化转型的智能外卖平台,基于SpringCloud微服务架构,解决校园餐厅排队拥堵、订单管理效率低等问题,支持学生小程序预订取餐与餐厅网页端智能管理,集成本地大模型提供智能辅助功能。}
    \begin{normalsize}
      \begin{itemize}
        \item \textbf{项目功能}:学生小程序端支持预约点餐、在线支付、取餐提醒,餐厅网页端管理菜品、套餐、订单排程,且平台集成本地大语言模型,餐厅员工通过智能助手获取订单优化建议和库存查询。
        \item \textbf{相关技术}: Java/SpringBoot/SpringCloud/RabbitMQ/MyBatisPlus/MySQL/Redis/Ollama/Vue.js
        \item \textbf{个人负责}:设计高可用微服务架构,主导核心服务开发(订单/支付/菜品/大模型服务),完成Ollama大模型本地化部署和接入项目,并完成员工网页端前端模块开发(菜品管理/员工管理/销售报表)。
        \item \textbf{技术亮点}:
        \setlength{\itemindent}{1em} % 子项缩进调整
          \item[$\circ$] \textbf{Spring后端与数据管理}:基于Spring Boot+MyBatisPlus实现高效数据访问,MySQL主从分离优化订单与库存查询性能,单表日均处理订单数据10万+条,查询响应时间降至50ms,显著提升系统稳定性与扩展性。
          \item[$\circ$] \textbf{微服务架构设计}:采用Spring Boot分层模式和Spring Cloud组件(Nacos、Gateway),将订单、用户、菜品等模块拆分为独立服务,提升代码复用率和可维护性。
          \item[$\circ$] \textbf{高并发优化}:使用Redis主从+哨兵模式缓存菜品信息,查询时间从600ms降至70ms,集成RabbitMQ实现异步解耦,系统吞吐量提升50\%,支撑10万+并发请求。
          \item[$\circ$] \textbf{智能化与认证}:集成Ollama本地大模型通过SpringAI提供智能订单处理,提升出餐效率30\%;实现微信小程序登录与JWT认证,支持日均300+次认证请求。
          \item[$\circ$] \textbf{数据可视化与自动化}:基于Vue.js+Echarts构建报表模块,实现订单量、销量等数据可视化,提升运营决策效率20\%。
        \end{itemize}
    \end{normalsize} % 引入Java项目经历文件

\subsection{\textbf{高校学生二手交易场景下的电子商务平台} \hspace{2cm}{2024.10 -- 2024.12}}
\fontsize{11pt}{0}{\textbf{项目描述}:面向高校学生二手交易场景,基于\textbf{Go+Gin}开发,支持\textbf{物品发布、搜索、在线交易与管理},集成\textbf{ElasticSearch}提升检索效率,助力学生便捷获取学习生活用品。}
  \begin{normalsize}
    \begin{itemize}
  \item \textbf{个人负责}:负责系统架构设计与核心业务模块开发,基于\textbf{Gin+Gorm}实现商品、订单、用户管理等服务,集成\textbf{ElasticSearch搜索引擎}优化商品检索性能,并使用\textbf{React+Ant Design}完成前端界面开发。
  \item \textbf{技术亮点}:
    \setlength{\itemindent}{1em} % 子项缩进调整
  \item[$\circ$] \textbf{高性能后端架构}:采用\textbf{Gin框架分层设计}(路由层→逻辑层→数据访问层),结合\textbf{Goroutine}实现高并发异步处理,系统具备良好\textbf{扩展性与稳定性},接口响应时间降至\textbf{60ms},支持\textbf{大规模并发访问},并集成\textbf{JWT鉴权中间件},保障\textbf{数据安全与访问权限}。
  \item[$\circ$] \textbf{高效数据存储}:基于\textbf{Gorm+MySQL}实现商品、订单、用户等核心数据的高效读写与复杂事务处理,结合\textbf{索引优化与数据校验机制},单表日均处理数据\textbf{5万+条},显著提升\textbf{数据一致性、查询性能与系统扩展能力}。
  \item[$\circ$] \textbf{搜索引擎优化}:集成\textbf{ElasticSearch}实现多条件商品搜索(品类、校区、价格区间),响应时间从\textbf{300ms优化至60ms},有效提升\textbf{校园二手交易匹配效率},支撑日均\textbf{3000+次安全访问}。
  \item[$\circ$] \textbf{用户实时聊天}:基于\textbf{WebSocket}在\textbf{Gin}实现交易沟通,维护连接队列管理活跃连接,峰值并发达\textbf{2k+},结合\textbf{JWT}和\textbf{Redis}同步在线状态,消息异步持久化到\textbf{MySQL}并由\textbf{Goroutine}写入,支持心跳、重连与有序投递,峰值吞吐约\textbf{5k 消息/s},日均持久化\textbf{50k+条}。
  \item[$\circ$] \textbf{前端界面开发}:使用\textbf{React+Ant Design}构建管理端和学生端界面,实现商品、订单、用户等模块的高效管理,优化\textbf{表单、表格与交互流程},页面操作效率提升\textbf{40\%},显著提升\textbf{学生操作体验和系统易用性}。
    \setlength{\itemindent}{0em} % 子项缩进调整
    \item \textbf{相关技术}: Go/Gin/Gorm/MySQL/ElasticSearch/React/Ant Design
    \end{itemize}
  \end{normalsize} % 引入Go项目经历文件

\subsection{\textbf{教师管理平台} \hspace{2cm}{2024.05 -- 2024.06}}
\fontsize{11pt}{0}{\textbf{项目描述}:教师管理平台,实现对教师信息、课程安排、考勤记录等的综合管理,提升学校的人事管理效率。}
    \begin{normalsize}
      \begin{itemize}
        \item \textbf{项目功能}:管理员可以添加、编辑、删除教师信息,支持按部门、职称等条件查询,而且能够制定和调整教师的课程安排,记录教师的出勤情况,生成考勤报表。
        \item \textbf{相关技术}: Node.js/Express/MySQL/MongoDB/Mongoose/Vue.js/Element UI
        \item \textbf{个人负责}:负责后端服务架构设计与核心模块开发,基于Express框架实现教师信息管理、课程编排与考勤统计模块,设计并优化MySQL和MongoDB数据库结构,并使用Element UI组件完成部分前端页面模块的实现。
        \item \textbf{技术亮点}:
        \setlength{\itemindent}{1em} % 子项缩进调整
          \item[$\circ$] \textbf{后端架构}:采用Express的路由-控制器-服务层结构划分,结合中间件实现功能解耦,将教师信息、课程安排与考勤模块清晰拆分,提升代码可维护性和扩展性。
          \item[$\circ$] \textbf{数据库优化}:对教师工号、部门ID、课程时间等字段建立复合索引,将教师信息综合查询响应时间从1.8s缩短至0.4s。
          \item[$\circ$] \textbf{MongoDB日志记录}:采用MongoDB的BSON文档结构,结合Mongoose进行数据建模与读写操作,用于记录用户关键操作行为(如教师信息修改、考勤打卡等),突破传统关系型数据库的字段限制,支持日均200万+行为日志写入。
          \item[$\circ$] \textbf{非阻塞 I/O}:利用Node.js的事件驱动与非阻塞I/O模型,在高并发场景下保持接口响应的高可用性和吞吐量,系统在模拟并发1000+请求时仍保持稳定运行,无明显性能瓶颈。
          \item[$\circ$] \textbf{JWT鉴权}:基于自定义中间件与express-jwt,结合async-local-storage实现用户认证与权限控制,权限校验平均耗时控制在 200ms 以内,支持日均 3000+ 次安全访问请求。
        \end{itemize}
    \end{normalsize} % 引入Node.js项目经历文件

\subsection{\textbf{技术团队协作场景下的在线博客系统} \hspace{2cm}{2024.07 -- 2024.09}}
\fontsize{11pt}{0}{\textbf{项目描述}:面向IT技术团队的知识分享与协作博客平台,基于FastAPI的前后端分离架构设计,专门服务于企业内部技术文档管理、项目经验总结、代码片段共享等场景,采用异步架构设计,具备文章管理、团队协作、评论系统、标签分类等综合博客功能。}
  \begin{normalsize}
    \begin{itemize}
    \item \textbf{项目功能}:支持技术文档发布编辑、项目复盘记录、代码片段分享,支持按技术栈、项目类型、时间等条件查询,团队成员可进行技术讨论互动,统计知识库阅读量和团队活跃度等。
    \item \textbf{相关技术}: Python/FastAPI/SQLAlchemy/MySQL/Vue.js/Element UI
    \item \textbf{个人负责}:负责后端服务架构设计与核心模块开发,基于FastAPI实现技术文档管理、团队协作系统与讨论交互模块,设计并优化MySQL数据库结构,并使用Vue.js和Element UI完成前端页面模块的实现。
    \item \textbf{技术亮点}:
    \setlength{\itemindent}{1em} % 子项缩进调整
      \item[$\circ$] \textbf{后端架构与数据库优化}:采用FastAPI路由-控制层-服务层结构,使用SQLAlchemy ORM建立复合索引,优化查询逻辑与连接池配置,文档检索响应时间从1.8s缩短至0.4s。
      \item[$\circ$] \textbf{高并发与安全认证}:利用FastAPI异步特性和Uvicorn服务器,支持1000+并发请求稳定运行;集成JWT中间件实现权限控制,权限校验平均耗时200ms,支持日均3000+次安全访问。
      \item[$\circ$] \textbf{智能推荐与数据分析}:集成Pandas+Scikit-learn实现用户行为分析,使用TF-IDF和余弦相似度算法构建个性化推荐系统,推荐准确率达75\%以上,日均处理分析数据10万+条。
      \item[$\circ$] \textbf{前端UI与数据可视化}:基于Vue.js+Element UI构建高可用博客界面,集成Matplotlib+ECharts.js实现阅读趋势、用户活跃度等数据可视化展示,提升内容运营效率。
    \end{itemize}
  \end{normalsize} % 引入Python项目经历文件

\subsection{\textbf{图像风格迁移平台} \hspace{2cm} {2024.09 -- 2024.11}}
\fontsize{11pt}{0}{\textbf{项目描述}:基于Flask的Web应用,集成VGG19神经网络,实现用户上传内容图和风格图后,自动完成图像风格迁移,生成融合两者特征的艺术作品。}
\begin{normalsize}
  \begin{itemize}
    \item \textbf{项目功能}:支持用户在线上传内容图与风格图,系统自动融合生成风格化新图像,并可在线预览与下载,提升普通用户的AI艺术体验。
    \item \textbf{相关技术}:Python/Flask/VGG19/PyTorch
    \item \textbf{个人负责}:独立完成Web端整体开发,基于Flask实现前后端交互、文件上传与结果展示,负责VGG19模型的迁移学习、推理流程及性能优化。
    \item \textbf{技术亮点}:
      \setlength{\itemindent}{1em}
      \item[$\circ$] \textbf{深度学习集成}:利用PyTorch加载预训练VGG19模型,优化风格迁移算法,提升生成图像的艺术表现力和细节还原度。
      \item[$\circ$] \textbf{高效推理}:通过GPU加速与异步任务队列,单次风格迁移平均耗时缩短至2.1s,支持多用户并发访问。
      \item[$\circ$] \textbf{前后端分离}:采用AJAX实现图片上传与结果异步刷新,提升用户体验,前端界面美观简洁,支持移动端自适应。
      \item[$\circ$] \textbf{安全与容错}:实现上传文件类型校验与异常处理,防止恶意文件上传,系统稳定性高。
  \end{itemize}
\end{normalsize}

\section{荣誉奖项}
% \datedline{\textit{\nth{1} Prize}, Award on xxx }{Jun. 2013}
% \datedline{Other awards}{2015}
\begin{normalsize}
  \begin{itemize}[parsep=0.5ex]
    \item 2023-2024年,华南师范大学本科生优秀奖学金\textbf{三等奖}
    \item 2024年,第15届蓝桥杯广东赛区Python程序设计A组\textbf{三等奖}
    \item 2025年,团体程序设计天梯赛\textbf{成功参赛奖}
    \item 2024年,第6届马蹄杯全国大学生程序设计本科赛道国赛\textbf{铜奖}
    \item 2023年,全国大学生数学建模竞赛\textbf{优胜奖}
  \end{itemize}
\end{normalsize} % 引入荣誉奖项文件

% \section{自我评价}
% \begin{itemize}
%   \begin{normalsize}
%       \item 简历的制作过程考验了一个人的两个能力,逻辑能力和细节能力。写好一份简历,有很多技巧,排版,量化数据等,但有一点最重要的是,自身要有实力。
%   \end{normalsize}
% \end{itemize}


\end{document}
