% !TEX program = xelatex

\documentclass{resume}
\usepackage{graphicx}
\usepackage{tabu}
\usepackage{multirow}
\usepackage[utf8]{inputenc}  
\usepackage{zh_CN-Adobefonts_external} % 
\usepackage{linespacing_fix} % disable extra space before next section
\usepackage{cite}
\usepackage{subfig}   %并排图片宏包
\usepackage{float}    %图片浮动位置


\begin{document}
\pagenumbering{gobble} % suppress displaying page number

\begin{minipage}{0.8\textwidth}
 \fontsize{11pt}{0}{
  姓名:\textbf{褚世超} \hspace{3.05cm} 年龄:\textbf{21}  \\
  性别:\textbf{男} \hspace{3.85cm} 求职意向:\textbf{后端开发}  \\
  联系电话:\textbf{13642563270} \hspace{1.35cm}  邮箱:\textbf{hongch666@gmail.com}
 }
  % \begin{normalsize}
  %   \begin{itemize}[parsep=0.5ex]
  %     \item 数学:离散数学、高等xxxx、高等数学、高等xxxx、高等数学
  %     \item 计算机:计算机xxxx、计算机xxxx(ML)、计算机xxxx(NLP)、计算机xxxx(KG)、计算机xxxx(DM)、数据结构、计算机xxxx、计算机xxxx、计算机网络
  %   \end{itemize}
  % \end{normalsize}
\end{minipage}
\begin{minipage}{0.2\textwidth}
\centering
\vspace{-0.4in} % 向上移动图片
\includegraphics[height=1.0in,]{pic/pic.jpg}
\end{minipage}

\section{教育背景}
\begin{normalsize}
  \begin{itemize}[parsep=0.5ex]
    \item {2022.09 -- 至今}\hspace{0.8cm}华南师范大学(211)  \hspace{0.8cm} 计算机科学与技术 \hspace{0.8cm} 学士 \hspace{0.8cm} GPA:3.61(前15\%)
  \end{itemize}
\end{normalsize}


\section{专业技能}
\begin{normalsize}
  \begin{itemize}
    \item \textbf{Java基础}:熟悉 Java 基础语法、集合框架与 JVM 内存模型,掌握并发编程与垃圾回收机制,理解面向对象设计原则。
    \item \textbf{数据库}:具备MySQL等关系型数据库设计与优化能力,熟悉事务管理和索引优化;掌握Redis、MongoDB、ElasticSearch等NoSQL数据库或搜索引擎的应用,能够根据需求灵活选择数据存储方案,提升查询效率。
    \item \textbf{Spring后端}:熟悉Maven构建与依赖管理,熟练使用Spring生态(SpringBoot、SpringMVC等)构建后端服务,掌握MyBatis/MyBatisPlus持久层框架,理解依赖注入(DI)和面向切面编程(AOP)机制,具备基于RESTful接口开发能力,具备前后端分离开发经验。
    \item \textbf{微服务}:熟悉SpringCloud微服务架构,熟悉Nacos服务注册与发现、Spring Cloud LoadBalancer负载均衡、OpenFeign服务调用、Sentinel服务保护、Gateway服务网关;具备使用RabbitMQ实现异步解耦、消息确认与异常处理的能力,支持高并发业务场景。
    \item \textbf{Go后端}:熟悉Go语言基础语法,掌握Goroutine与Channel并发模型;具备基于Gin框架的 RESTful接口开发能力,能够使用GORM进行数据库操作,完成高性能Web服务的开发与部署。
    \item \textbf{Node.js后端}:熟悉Node.js及主流框架(如Express、NestJS),具备模块化后端开发能力,能够使用Mongoose操作MongoDB,熟练用TypeORM构建关系型数据模型并支持事务控制。
    \item \textbf{Python后端}:熟悉Python语言基础与异步编程,具备基于FastAPI框架的RESTful接口开发能力,能够使用SQLAlchemy ORM进行数据库操作,完成高性能Web服务的开发与部署;能够利用Python的高效库(如Pandas、NumPy等)进行数据处理、分析与可视化。
    \item \textbf{AI技能}:具备本地大模型部署经验,能够使用Ollama部署大模型并结合SpringAI构建AI应用;熟悉数据结构与算法,了解传统机器学习及深度学习算法。
    \item \textbf{前端技术}:了解Javascript/Typescript等前端技术,了解Vue.js核心概念及Element UI组件库,能够高效构建前端项目;了解React框架及Ant DesignUI框架。
    \item \textbf{开发环境管理}:熟悉使用Git进行版本控制的流程;熟悉Linux操作系统的基本命令,能够在Linux环境下进行开发和部署;熟悉Docker容器化部署,能够进行高效服务集群的搭建。
    
  \end{itemize}
\end{normalsize}

\section{实习经历}
\datedsubsection{\textbf{广州拓扑网络技术有限公司}\hspace{2.5cm}后端开发实习生}{2025.05 -- 2025.08}
\fontsize{11pt}{0}{\textbf{实习描述}:参与基于NestJS的企业级RPA管理平台开发,负责实现机器人、任务及调度模块后端逻辑,平台支持对影刀(Yingdao)RPA机器人的生命周期管理、任务配置与高并发调度。}
    \begin{normalsize}
      \begin{itemize}
        \item \textbf{核心职责}:
        \setlength{\itemindent}{1em} % 子项缩进调整
          \item[$\circ$] \textbf{后端架构}:参与后端接口开发,基于NestJS框架实现机器人、任务、调度等模块接口,完成30+个接口的编写与联调。
          \item[$\circ$] \textbf{非阻塞I/O}:充分利用Node.js异步事件驱动与非阻塞I/O模型,在任务高并发调度与状态修改场景中保持系统高吞吐与低延迟,稳定支撑千级并发访问请求。
          \item[$\circ$] \textbf{数据库框架}:使用TypeORM构建数据模型与多表关联关系,优化查询逻辑,提升机器人任务检索性能约40\%。
          \item[$\circ$] \textbf{MongoDB日志记录}:使用MongoDB结合Mongoose建模,记录机器人执行任务关键行为(如采集商品、上架商品等),支持日均2万+操作日志高效写入,突破关系型数据库字段限制。
          \item[$\circ$] \textbf{Redis缓存}:实现机器人状态缓存和任务缓存,接口查询响应时间由原500ms降低至 20ms,支撑10万级高并发状态请求。
          \item[$\circ$] \textbf{RabbitMQ异步解耦​​}:负责集成RabbitMQ,实现任务异步投递与空闲机器人动态分配,系统日均调度任务数超5000条,任务分发延迟控制在200ms内。
          \item[$\circ$] \textbf{JWT鉴权}:实现基于JWT的登录鉴权,并结合自定义角色守卫与权限守卫进行接口访问控制。
          % \item[$\circ$] \textbf{接口文档}:基于Swagger自动生成接口文档,覆盖率达100\%,显著提升前后端联调效率,减少沟通成本约30\%。
        \setlength{\itemindent}{0em} % 子项缩进调整
        \item \textbf{相关技术}: NestJS/TypeORM/MySQL/Redis/MongoDB/RabbitMQ/JWT
        \end{itemize}
    \end{normalsize}


\section{项目经历}

\subsection{\textbf{外卖点单和管理平台} \hspace{2cm}{2024.12 -- 2025.04}}
\fontsize{11pt}{0}{\textbf{项目描述}:采用前后端分离的技术,实现用户在小程序上下单,店铺员工通过浏览器进行菜品、套餐和订单等的管理和配送,且员工在网页端可以使用本地大模型聊天功能辅助管理对应的相关信息。}
    \begin{normalsize}
      \begin{itemize}
        \item \textbf{项目功能}:用户通过小程序完成下单、支付和订单查询,商家员工则在网页端管理菜品、套餐和订单等信息,​平台还集成本地部署的大语言模型,员工可通过聊天界面获取管理建议和信息查询,提升运营效率。
        \item \textbf{相关技术}: SpringBoot/SpringCloud/RabbitMQ/MyBatisPlus/MySQL/Redis/Ollama/Vue.js
        \item \textbf{个人负责}:设计高可用微服务架构,主导核心服务开发(订单/支付/菜品/大模型服务),完成Ollama大模型本地化部署和接入项目,并完成员工网页端前端模块开发(菜品管理/员工管理/销售报表)。
        \item \textbf{技术亮点}:
        \setlength{\itemindent}{1em} % 子项缩进调整
          \item[$\circ$] \textbf{后端架构}:采用Spring Boot的Controller-Service-DAO分层模式,将员工信息管理、菜品管理、订单管理等模块解耦,提升代码复用率和可维护性。
          \item[$\circ$] \textbf{微服务架构升级}:采用Spring Cloud组件(如Nacos服务注册发现、Gateway网关等),将订单、用户、菜品管理等模块拆分为独立服务。
          \item[$\circ$] \textbf{RabbitMQ异步解耦​​}:通过RabbitMQ实现微服务之间的异步调用,将订单创建、支付处理等操作解耦,提升系统吞吐量50\%。
          \item[$\circ$] \textbf{Redis缓存}:使用主从+哨兵模式进行Redis的部署,将菜品信息与套餐信息进行缓存,使用户在小程查询时间从600ms降至70ms,支撑10万+的并发请求。
          % \item[$\circ$] \textbf{JWT鉴权}:基于自定义拦截器和ThreadLocal实现员工的认证与鉴权,实现200ms内完成权限校验,支撑日均3000+次的安全访问。
          \item[$\circ$] \textbf{微信登录集成}:在小程序端集成微信登录功能,通过获取code调用微信开放平台接口实现用户身份验证,并结合后端JWT令牌机制与自定义拦截器完成登录校验与用户信息绑定,支持日均300+ 次稳定认证请求。
          \item[$\circ$] \textbf{前端Echarts 报表展示}:在员工端集成 Echarts 图表库,支持对近30日订单量、菜品销量等数据进行可视化展示,提供日均订单趋势图、品类销售分布图等分析报表,提升商家对运营数据的洞察力,辅助实现约20\%的运营决策效率提升。
          \item[$\circ$] \textbf{本地大模型接入}:使用Ollama进行本地大模型的部署,并通过SpringAI的Ollama依赖进行后端API调用,实现员工智能处理订单请求,提升约30\%的出餐效率。
        \end{itemize}
    \end{normalsize}

\subsection{\textbf{电子商务平台} \hspace{2cm}{2024.10 -- 2024.12}}
\fontsize{11pt}{0}{\textbf{项目描述}:采用前后端分离架构,实现用户在线购物和商家后台管理,包括用户、商品、订单、分类等模块的统一管理。}
  \begin{normalsize}
    \begin{itemize}
    \item \textbf{项目功能}:电子商务平台,支持用户下单、搜索商品、查看订单等功能,商家可通过后台对商品信息和订单进行管理。
    \item \textbf{相关技术}: Go/Gin/Gorm/MySQL/ElasticSearch/React/Ant Design
    \item \textbf{个人负责}:采用Gin+Gorm构建业务模块(商品中心/订单中心/用户中心/搜索服务),实现ElasticSearch搜索集群并接入项目的搜索服务,优化搜索接口性能,并负责管理端和用户端的React项目开发,实现商品、订单、用户等模块的管理和用户商品浏览、下单等核心功能。
    \item \textbf{技术亮点}:
    \setlength{\itemindent}{1em} % 子项缩进调整
      \item[$\circ$] \textbf{后端架构}:基于Gin框架的轻量级高性能特性,采用分层架构设计(路由层→ 逻辑层→ 数据访问层),通过Goroutine实现异步非阻塞I/O模型,使API接口响应时间下降至60ms。
        \item[$\circ$] \textbf{JWT鉴权}:使用JWT鉴权中间件实现管理员和用户的认证与鉴权,实现200ms内完成权限校验,支撑日均3000+次的安全访问。
        \item[$\circ$] \textbf{ElasticSearch搜索优化}:集成ElasticSearch实现商品搜索功能,支持全文检索和模糊匹配,将响应时间从300ms优化至60ms,有效提升搜索结果的匹配精度。
        \item[$\circ$] \textbf{前端页面}:使用React进行管理端和用户端开发,基于Ant Design组件库实现商品、订单、用户等模块的高效管理界面,优化表单、表格与交互流程,提升用户操作体验和系统易用性。
    \end{itemize}
  \end{normalsize}

\subsection{\textbf{教师管理平台} \hspace{2cm}{2024.07 -- 2024.09}}
\fontsize{11pt}{0}{\textbf{项目描述}:教师管理平台,实现对教师信息、课程安排、考勤记录等的综合管理,提升学校的人事管理效率。}
  \begin{normalsize}
    \begin{itemize}
    \item \textbf{项目功能}:管理员可以添加、编辑、删除教师信息,支持按部门、职称等条件查询,而且能够制定和调整教师的课程安排,记录教师的出勤情况等。
    \item \textbf{相关技术}: Python/FastAPI/SQLAlchemy/MySQL/React/Ant Design
    \item \textbf{个人负责}:负责后端服务架构设计与核心模块开发,基于FastAPI实现教师信息管理、课程编排与考勤统计模块,设计并优化MySQL数据库结构,并使用React和Ant Design完成部分前端页面模块的实现。
    \item \textbf{技术亮点}:
    \setlength{\itemindent}{1em} % 子项缩进调整
      \item[$\circ$] \textbf{后端架构}:采用FastAPI路由-控制层-服务层结构划分,结合中间件实现功能解耦,将教师信息、课程安排与考勤模块清晰拆分,提升代码可维护性和扩展性。
      \item[$\circ$] \textbf{数据库优化}:对教师工号、部门ID、课程时间等字段建立复合索引,将教师信息综合查询响应时间从1.8s缩短至0.4s。
      \item[$\circ$] \textbf{异步高并发}:利用FastAPI的异步特性和Uvicorn服务器,在高并发场景下保持接口响应的高可用性和吞吐量,系统在模拟并发1000+请求时仍保持稳定运行,无明显性能瓶颈。
      \item[$\circ$] \textbf{JWT鉴权}:基于FastAPI中间件与PyJWT,实现用户认证与权限控制,权限校验平均耗时控制在200ms以内,支持日均3000+次安全访问请求。
      \item[$\circ$] \textbf{前端UI优化}:前端采用React结合Ant Design组件库,快速搭建高可用的管理后台,提升表单、表格、弹窗等交互体验,实现教师信息、课程安排等模块的高效可视化管理。
    \end{itemize}
  \end{normalsize}

% \subsection{{2024.06 -- 2024.07} \hspace{2cm}\textbf{教师管理平台}}
% \fontsize{11pt}{0}{\textbf{项目描述}:教师管理平台,实现对教师信息、课程安排、考勤记录等的综合管理,提升学校的人事管理效率。}
%     \begin{normalsize}
%       \begin{itemize}
%         \item \textbf{项目功能}:管理员可以添加、编辑、删除教师信息,支持按部门、职称等条件查询,而且能够制定和调整教师的课程安排,记录教师的出勤情况,生成考勤报表。
%         \item \textbf{相关技术}: Node.js/Express/MySQL/MongoDB/Mongoose/Vue.js/Element UI
%         \item \textbf{个人负责}:负责后端服务架构设计与核心模块开发,基于Express框架实现教师信息管理、课程编排与考勤统计模块,设计并优化MySQL和MongoDB数据库结构,并使用Element UI组件完成部分前端页面模块的实现。
%         \item \textbf{技术亮点}:
%         \setlength{\itemindent}{1em} % 子项缩进调整
%           \item[$\circ$] \textbf{后端架构}:采用Express的路由-控制器-服务层结构划分,结合中间件实现功能解耦,将教师信息、课程安排与考勤模块清晰拆分,提升代码可维护性和扩展性。
%           \item[$\circ$] \textbf{数据库优化}:对教师工号、部门ID、课程时间等字段建立复合索引,将教师信息综合查询响应时间从1.8s缩短至0.4s。
%           \item[$\circ$] \textbf{MongoDB日志记录}:采用MongoDB的BSON文档结构,结合Mongoose进行数据建模与读写操作,用于记录用户关键操作行为(如教师信息修改、考勤打卡等),突破传统关系型数据库的字段限制,支持日均200万+行为日志写入。
%           \item[$\circ$] \textbf{非阻塞 I/O}:利用Node.js的事件驱动与非阻塞I/O模型,在高并发场景下保持接口响应的高可用性和吞吐量,系统在模拟并发1000+请求时仍保持稳定运行,无明显性能瓶颈。
%           \item[$\circ$] \textbf{JWT鉴权}:基于自定义中间件与express-jwt,结合async-local-storage实现用户认证与权限控制,权限校验平均耗时控制在 200ms 以内,支持日均 3000+ 次安全访问请求。
%         \end{itemize}
%     \end{normalsize}

% \subsection{{2024.09 -- 2024.11} \hspace{2cm}\textbf{图像风格迁移项目}}
% \fontsize{11pt}{0}{项目描述:一个基于Flask的Web应用,利用VGG19神经网络模型,实现用户上传内容图和风格图后,进行图像风格迁移,生成融合两者特征的全新图像。}
%     \begin{normalsize}
%       \begin{itemize}
%         \item 项目功能:用户通过网站上传内容图和风格图,系统将两者融合,生成具有特定艺术风格的图像。
%         \item 个人负责:负责整个网站的开发,包括使用Flask框架构建Web应用,以及部分实现图像风格迁移模型的训练和推理。
%         \item 相关技术: VGG19/flask
%         \end{itemize}
%     \end{normalsize}

\section{荣誉奖项}
% \datedline{\textit{\nth{1} Prize}, Award on xxx }{Jun. 2013}
% \datedline{Other awards}{2015}
\begin{normalsize}
  \begin{itemize}[parsep=0.5ex]
    \item 2023-2024年,华南师范大学本科生优秀奖学金三等奖
    \item 2024年,第15届蓝桥杯广东赛区Python程序设计A组三等奖
    \item 2023年,全国大学生数学建模竞赛优胜奖
  \end{itemize}
\end{normalsize}

% \section{自我评价}
% \begin{itemize}
%   \begin{normalsize}
%       \item 简历的制作过程考验了一个人的两个能力,逻辑能力和细节能力。写好一份简历,有很多技巧,排版,量化数据等,但有一点最重要的是,自身要有实力。
%   \end{normalsize}
% \end{itemize}


\end{document}
