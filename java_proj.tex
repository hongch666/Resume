\subsection{\textbf{校园餐厅智能化场景下的外卖点单和管理平台} \hspace{2cm}{2025.01 -- 2025.04}}
\fontsize{11pt}{0}{\textbf{项目描述}:面向高校食堂数字化转型,打造基于\textbf{SpringCloud微服务}的智能外卖平台,支持\textbf{小程序点餐}、\textbf{在线支付}、\textbf{智能助手辅助管理},提升餐厅运营效率与用户体验。}
    \begin{normalsize}
      \begin{itemize}
        \item \textbf{个人负责}:设计\textbf{高可用微服务架构},主导\textbf{核心服务开发}(订单/支付/菜品/大模型服务),完成\textbf{Ollama大模型}本地化部署和接入项目,并完成员工网页端\textbf{前端模块开发}(菜品管理/员工管理/销售报表)。
        \item \textbf{技术亮点}:
        \setlength{\itemindent}{1em} % 子项缩进调整
          \item[$\circ$] \textbf{Spring后端与数据管理}:基于\textbf{Spring Boot+MyBatisPlus}实现高效数据访问,\textbf{MySQL主从分离}优化订单与库存查询性能,单表日均处理订单数据\textbf{10万+条},查询响应时间降至\textbf{50ms},显著提升系统\textbf{稳定性与扩展性}。
          \item[$\circ$] \textbf{微服务架构设计}:采用\textbf{Spring Boot分层模式}和\textbf{Spring Cloud组件}(Nacos、Gateway),将订单、用户、菜品等模块拆分为\textbf{独立服务},提升\textbf{代码复用率和可维护性}。
          \item[$\circ$] \textbf{高并发优化}:使用\textbf{Redis主从+哨兵模式}缓存菜品信息,查询时间从\textbf{600ms降至70ms},集成\textbf{RabbitMQ}实现异步解耦,系统吞吐量提升\textbf{50\%},支撑\textbf{10万+并发请求}。
          \item[$\circ$] \textbf{智能化与认证}:集成\textbf{Ollama本地大模型}通过\textbf{SpringAI}提供智能订单处理,提升\textbf{出餐效率30\%};实现\textbf{微信小程序登录}与\textbf{JWT认证},支持日均\textbf{300+次认证请求}。
          \item[$\circ$] \textbf{数据可视化与自动化}:基于\textbf{Vue.js+Echarts}构建报表模块,实现\textbf{订单量、销量等数据可视化},提升\textbf{运营决策效率20\%}。
        \setlength{\itemindent}{0em} % 子项缩进调整
        \item \textbf{相关技术}: Java/SpringBoot/SpringCloud/RabbitMQ/MyBatisPlus/MySQL/Redis/Ollama/Vue.js
        \end{itemize}
    \end{normalsize}