\subsection{\textbf{校园餐厅智能化场景下的外卖点单和管理平台} \hspace{2cm}{2025.01 -- 2025.04}}
\fontsize{11pt}{0}{\textbf{项目描述}:面向高校食堂数字化转型的智能外卖平台,基于SpringCloud微服务架构,解决校园餐厅排队拥堵、订单管理效率低等问题,支持学生小程序预订取餐与餐厅网页端智能管理,集成本地大模型提供智能辅助功能。}
    \begin{normalsize}
      \begin{itemize}
        \item \textbf{项目功能}:学生小程序端支持预约点餐、在线支付、取餐提醒,餐厅网页端管理菜品、套餐、订单排程,且平台集成本地大语言模型,餐厅员工通过智能助手获取订单优化建议和库存查询。
        \item \textbf{相关技术}: Java/SpringBoot/SpringCloud/RabbitMQ/MyBatisPlus/MySQL/Redis/Ollama/Vue.js
        \item \textbf{个人负责}:设计高可用微服务架构,主导核心服务开发(订单/支付/菜品/大模型服务),完成Ollama大模型本地化部署和接入项目,并完成员工网页端前端模块开发(菜品管理/员工管理/销售报表)。
        \item \textbf{技术亮点}:
        \setlength{\itemindent}{1em} % 子项缩进调整
          \item[$\circ$] \textbf{Spring后端与数据管理}:基于Spring Boot+MyBatisPlus实现高效数据访问,MySQL主从分离优化订单与库存查询性能,单表日均处理订单数据10万+条,查询响应时间降至50ms,显著提升系统稳定性与扩展性。
          \item[$\circ$] \textbf{微服务架构设计}:采用Spring Boot分层模式和Spring Cloud组件(Nacos、Gateway),将订单、用户、菜品等模块拆分为独立服务,提升代码复用率和可维护性。
          \item[$\circ$] \textbf{高并发优化}:使用Redis主从+哨兵模式缓存菜品信息,查询时间从600ms降至70ms,集成RabbitMQ实现异步解耦,系统吞吐量提升50\%,支撑10万+并发请求。
          \item[$\circ$] \textbf{智能化与认证}:集成Ollama本地大模型通过SpringAI提供智能订单处理,提升出餐效率30\%;实现微信小程序登录与JWT认证,支持日均300+次认证请求。
          \item[$\circ$] \textbf{数据可视化与自动化}:基于Vue.js+Echarts构建报表模块,实现订单量、销量等数据可视化,提升运营决策效率20\%。
        \end{itemize}
    \end{normalsize}