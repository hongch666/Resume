\subsection{\textbf{校园餐厅智能化场景下的外卖点单和管理平台} \hspace{2cm}{2025.01 -- 2025.04}}
\fontsize{11pt}{0}{\textbf{项目描述}:面向高校食堂数字化转型的智能外卖平台,基于SpringBoot微服务架构,专门解决校园餐厅排队拥堵、订单管理效率低等问题,支持学生小程序预订取餐与餐厅网页端智能管理,集成本地大模型提供智能辅助功能。}
    \begin{normalsize}
      \begin{itemize}
        \item \textbf{项目功能}:学生小程序端支持预约点餐、在线支付、取餐提醒,餐厅网页端管理菜品、套餐、订单排程,且平台集成本地大语言模型,餐厅员工通过智能助手获取订单优化建议和库存查询。
        \item \textbf{相关技术}: SpringBoot/SpringCloud/RabbitMQ/MyBatisPlus/MySQL/Redis/Ollama/Vue.js
        \item \textbf{个人负责}:设计高可用微服务架构,主导核心服务开发(订单/支付/菜品/大模型服务),完成Ollama大模型本地化部署和接入项目,并完成员工网页端前端模块开发(菜品管理/员工管理/销售报表)。
        \item \textbf{技术亮点}:
        \setlength{\itemindent}{1em} % 子项缩进调整
          \item[$\circ$] \textbf{后端架构}:采用Spring Boot的Controller-Service-DAO分层模式,将员工信息管理、菜品管理、订单管理等模块解耦,提升代码复用率和可维护性。
          \item[$\circ$] \textbf{微服务架构升级}:采用Spring Cloud组件(如Nacos服务注册发现、Gateway网关等),将订单、用户、菜品管理等模块拆分为独立服务。
          \item[$\circ$] \textbf{RabbitMQ异步解耦​​}:通过RabbitMQ实现微服务之间的异步调用,将订单创建、支付处理等操作解耦,提升系统吞吐量50\%。
          \item[$\circ$] \textbf{Redis缓存}:使用主从+哨兵模式进行Redis的部署,将菜品信息与套餐信息进行缓存,使用户在小程查询时间从600ms降至70ms,支撑10万+的并发请求。
          % \item[$\circ$] \textbf{JWT鉴权}:基于自定义拦截器和ThreadLocal实现员工的认证与鉴权,实现200ms内完成权限校验,支撑日均3000+次的安全访问。
          \item[$\circ$] \textbf{微信登录集成}:集成微信小程序登录,通过微信开放平台接口实现用户身份验证,结合JWT令牌与自定义拦截器完成登录校验,支持日均300+次认证请求。
          \item[$\circ$] \textbf{前端数据可视化}:基于Vue.js和Echarts图表库构建员工端报表模块,实现订单量、菜品销量等数据的可视化展示,提供趋势图、分布图等分析报表,提升运营决策效率约20\%。
          \item[$\circ$] \textbf{本地大模型接入}:使用Ollama进行本地大模型的部署,并通过SpringAI的Ollama依赖进行后端API调用,实现员工智能处理订单请求,提升约30\%的出餐效率。
        \end{itemize}
    \end{normalsize}