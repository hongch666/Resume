\subsection{\textbf{教师管理平台} \hspace{2cm}{2024.05 -- 2024.06}}
\fontsize{11pt}{0}{\textbf{项目描述}:教师管理平台,实现对教师信息、课程安排、考勤记录等的综合管理,提升学校的人事管理效率。}
    \begin{normalsize}
      \begin{itemize}
        \item \textbf{项目功能}:管理员可以添加、编辑、删除教师信息,支持按部门、职称等条件查询,而且能够制定和调整教师的课程安排,记录教师的出勤情况,生成考勤报表。
        \item \textbf{相关技术}: Node.js/Express/MySQL/MongoDB/Mongoose/Vue.js/Element UI
        \item \textbf{个人负责}:负责后端服务架构设计与核心模块开发,基于Express框架实现教师信息管理、课程编排与考勤统计模块,设计并优化MySQL和MongoDB数据库结构,并使用Element UI组件完成部分前端页面模块的实现。
        \item \textbf{技术亮点}:
        \setlength{\itemindent}{1em} % 子项缩进调整
          \item[$\circ$] \textbf{后端架构}:采用Express的路由-控制器-服务层结构划分,结合中间件实现功能解耦,将教师信息、课程安排与考勤模块清晰拆分,提升代码可维护性和扩展性。
          \item[$\circ$] \textbf{数据库优化}:对教师工号、部门ID、课程时间等字段建立复合索引,将教师信息综合查询响应时间从1.8s缩短至0.4s。
          \item[$\circ$] \textbf{MongoDB日志记录}:采用MongoDB的BSON文档结构,结合Mongoose进行数据建模与读写操作,用于记录用户关键操作行为(如教师信息修改、考勤打卡等),突破传统关系型数据库的字段限制,支持日均200万+行为日志写入。
          \item[$\circ$] \textbf{非阻塞 I/O}:利用Node.js的事件驱动与非阻塞I/O模型,在高并发场景下保持接口响应的高可用性和吞吐量,系统在模拟并发1000+请求时仍保持稳定运行,无明显性能瓶颈。
          \item[$\circ$] \textbf{JWT鉴权}:基于自定义中间件与express-jwt,结合async-local-storage实现用户认证与权限控制,权限校验平均耗时控制在 200ms 以内,支持日均 3000+ 次安全访问请求。
        \end{itemize}
    \end{normalsize}