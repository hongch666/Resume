\section{实习经历}
\datedsubsection{\textbf{广州拓扑网络技术有限公司}\hspace{2.5cm}后端开发实习生}{2025.05 -- 2025.08}
\fontsize{11pt}{0}{\textbf{实习描述}:参与基于NestJS的企业级内部管理平台开发,负责RPA机器人管理、AI知识库、逆向物流自动化等模块后端开发,集成影刀RPA、Coze AI、钉钉开放平台,实现业务流程自动化与智能化管理。}
    \begin{normalsize}
      \begin{itemize}
        \item \textbf{核心职责}:
        \setlength{\itemindent}{1em} % 子项缩进调整
          \item[$\circ$] \textbf{后端框架与接口开发}:基于NestJS实现机器人、任务、调度、逆向物流等模块接口,独立完成30+接口开发与联调,支持影刀RPA机器人生命周期及退换货、维修等业务自动化,相关处理效率提升60\%。
          \item[$\circ$] \textbf{Coze AI知识库集成}:接入Coze AI平台,构建企业知识库管理系统,实现文档自动上传、知识检索、智能问答等功能。
          \item[$\circ$] \textbf{钉钉与JWT统一认证}:对接钉钉开放平台,支持钉钉扫码登录与消息推送,结合JWT实现统一身份认证与权限控制,日均支撑500+用户登录与消息通知,保障系统安全与高效协同。
          \item[$\circ$] \textbf{非阻塞I/O}:利用Node.js异步事件驱动与非阻塞I/O模型,在任务高并发调度与状态修改场景中保持系统高吞吐与低延迟,稳定支撑千级并发访问请求。
          \item[$\circ$] \textbf{数据库框架}:使用TypeORM构建数据模型与多表关联关系,设计机器人、任务、用户等核心实体映射,通过Repository模式封装数据访问层,优化复杂查询逻辑与索引策略,提升性能约40\%。
          \item[$\circ$] \textbf{MongoDB日志记录}:使用MongoDB结合Mongoose建模,记录系统关键业务操作与用户行为日志,支持日均2万+操作日志高效写入,突破关系型数据库字段限制。
          \item[$\circ$] \textbf{Redis缓存}:实现机器人状态缓存和任务缓存,接口查询响应时间由原500ms降低至20ms,支撑10万级高并发状态请求。
          \item[$\circ$] \textbf{RabbitMQ异步解耦}:集成RabbitMQ,实现任务异步投递与空闲机器人动态分配,系统日均调度任务数超5000条,任务分发延迟控制在200ms内。
          % \item[$\circ$] \textbf{JWT鉴权}:实现基于JWT的登录鉴权,并结合自定义角色守卫与权限守卫进行接口访问控制。
          % \item[$\circ$] \textbf{接口文档}:基于Swagger自动生成接口文档,覆盖率达100\%,显著提升前后端联调效率,减少沟通成本约30\%。
        \setlength{\itemindent}{0em} % 子项缩进调整
        \item \textbf{相关技术}: NestJS/TypeORM/MySQL/Redis/MongoDB/RabbitMQ/JWT/Coze AI/钉钉开放平台/影刀RPA
        \end{itemize}
    \end{normalsize}