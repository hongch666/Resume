\section{实习经历}
\datedsubsection{\textbf{广州拓扑网络技术有限公司}\hspace{2.5cm}后端开发实习生}{2025.05 -- 2025.08}
\fontsize{11pt}{0}{\textbf{实习描述}:参与基于NestJS的企业级RPA管理平台开发,负责实现机器人、任务及调度模块后端逻辑,平台支持对影刀(Yingdao)RPA机器人的生命周期管理、任务配置与高并发调度。}
    \begin{normalsize}
      \begin{itemize}
        \item \textbf{核心职责}:
        \setlength{\itemindent}{1em} % 子项缩进调整
          \item[$\circ$] \textbf{后端架构}:参与后端接口开发,基于NestJS框架实现机器人、任务、调度等模块接口,完成30+个接口的编写与联调。
          \item[$\circ$] \textbf{非阻塞I/O}:充分利用Node.js异步事件驱动与非阻塞I/O模型,在任务高并发调度与状态修改场景中保持系统高吞吐与低延迟,稳定支撑千级并发访问请求。
          \item[$\circ$] \textbf{数据库框架}:使用TypeORM构建数据模型与多表关联关系,优化查询逻辑,提升机器人任务检索性能约40\%。
          \item[$\circ$] \textbf{MongoDB日志记录}:使用MongoDB结合Mongoose建模,记录机器人执行任务关键行为(如采集商品、上架商品等),支持日均2万+操作日志高效写入,突破关系型数据库字段限制。
          \item[$\circ$] \textbf{Redis缓存}:实现机器人状态缓存和任务缓存,接口查询响应时间由原500ms降低至 20ms,支撑10万级高并发状态请求。
          \item[$\circ$] \textbf{RabbitMQ异步解耦​​}:负责集成RabbitMQ,实现任务异步投递与空闲机器人动态分配,系统日均调度任务数超5000条,任务分发延迟控制在200ms内。
          \item[$\circ$] \textbf{JWT鉴权}:实现基于JWT的登录鉴权,并结合自定义角色守卫与权限守卫进行接口访问控制。
          % \item[$\circ$] \textbf{接口文档}:基于Swagger自动生成接口文档,覆盖率达100\%,显著提升前后端联调效率,减少沟通成本约30\%。
        \setlength{\itemindent}{0em} % 子项缩进调整
        \item \textbf{相关技术}: NestJS/TypeORM/MySQL/Redis/MongoDB/RabbitMQ/JWT
        \end{itemize}
    \end{normalsize}